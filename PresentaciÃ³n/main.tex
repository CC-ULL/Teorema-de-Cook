% Copyright 2004 by Till Tantau <tantau@users.sourceforge.net>.
%
% In principle, this file can be redistributed and/or modified under
% the terms of the GNU Public License, version 2.
%
% However, this file is supposed to be a template to be modified
% for your own needs. For this reason, if you use this file as a
% template and not specifically distribute it as part of a another
% package/program, I grant the extra permission to freely copy and
% modify this file as you see fit and even to delete this copyright
% notice. 

\documentclass[10pt, mathserif, profesionalfont]{beamer}

\usepackage[utf8]{inputenc}
\usepackage[spanish, es-tabla]{babel}

\usepackage[top=3cm,bottom=3cm,outer=2.0cm, inner=3.5cm, marginparwidth = 1.0cm, marginparsep=0.5cm,headsep=10pt, a4paper, heightrounded]{geometry}

\usepackage{amsmath,amsfonts,amssymb,amsthm}
\usepackage{booktabs}
\usepackage{hyperref}
\usepackage{authblk}

\usepackage{empheq}

\usepackage[acronym]{glossaries}

\renewcommand\Affilfont{\itshape\small}
\renewcommand\Authand{ y }

\newtheorem{thm}{Teorema}

\usepackage{array}
\newcolumntype{L}[1]{>{\raggedright\let\newline\\\arraybackslash\hspace{0pt}}p{#1}}
\newcolumntype{C}[1]{>{\centering\let\newline\\\arraybackslash\hspace{0pt}}p{#1}}
\newcolumntype{R}[1]{>{\raggedleft\let\newline\\\arraybackslash\hspace{0pt}}p{#1}}

\newcolumntype{X}[1]{>{\raggedright\let\newline\\\arraybackslash\hspace{0pt}}m{#1}}
\newcolumntype{Y}[1]{>{\centering\let\newline\\\arraybackslash\hspace{0pt}}m{#1}}
\newcolumntype{Z}[1]{>{\raggedleft\let\newline\\\arraybackslash\hspace{0pt}}m{#1}}

\newacronym{ndtm}{NDTM}{Máquina de Turing No Determinista}

\newacronym{sat}{SAT}{SATISFACTIBILIDAD}

\newacronym{3sat}{3SAT}{3-SATISFACTIBILIDAD}




% There are many different themes available for Beamer. A comprehensive
% list with examples is given here:
% http://deic.uab.es/~iblanes/beamer_gallery/index_by_theme.html
% You can uncomment the themes below if you would like to use a different
% one:
%\usetheme{AnnArbor}
%\usetheme{Antibes}
%\usetheme{Bergen}
%\usetheme{Berkeley}
%\usetheme{Berlin}
%\usetheme{Boadilla}
%\usetheme{boxes}
%\usetheme{CambridgeUS}
%\usetheme{Copenhagen}
%\usetheme{Darmstadt}
%\usetheme{default}
%\usetheme{Frankfurt}
%\usetheme{Goettingen}
%\usetheme{Hannover}
%\usetheme{Ilmenau}
%\usetheme{JuanLesPins}
%\usetheme{Luebeck}
\usetheme{Madrid}
%\usetheme{Malmoe}
%\usetheme{Marburg}
%\usetheme{Montpellier}
%\usetheme{PaloAlto}
%\usetheme{Pittsburgh}
%\usetheme{Rochester}
%\usetheme{Singapore}
%\usetheme{Szeged}
%\usetheme{Warsaw}

\title{3-SATISFACTIBILIDAD}

% A subtitle is optional and this may be deleted
\subtitle{Seis problemas básicos $\mathcal{NP}$-completos}

\author{Adrián Rodríguez Bazaga, Eleazar Díaz Delgado, Rudolf Cicko}
% - Give the names in the same order as the appear in the paper.
% - Use the \inst{?} command only if the authors have different
%   affiliation.

\institute[Universidad de La Laguna]{Adrián Rodríguez Bazaga, Eleazar Díaz Delgado \and Rudolf Cicko} % (optional, but mostly needed)

% - Use the \inst command only if there are several affiliations.
% - Keep it simple, no one is interested in your street address.

\date{Complejidad Computacional, Grado en Ingeniería Informática \newline Universidad de La Laguna \newline Curso 2016-2017}
% - Either use conference name or its abbreviation.
% - Not really informative to the audience, more for people (including
%   yourself) who are reading the slides online

\subject{Theoretical Computer Science}
% This is only inserted into the PDF information catalog. Can be left
% out. 

% If you have a file called "university-logo-filename.xxx", where xxx
% is a graphic format that can be processed by latex or pdflatex,
% resp., then you can add a logo as follows:

% \pgfdeclareimage[height=0.5cm]{university-logo}{university-logo-filename}
% \logo{\pgfuseimage{university-logo}}

% Delete this, if you do not want the table of contents to pop up at
% the beginning of each subsection:
\AtBeginSubsection[]
{
  \begin{frame}<beamer>{Outline}
    \tableofcontents[currentsection,currentsubsection]
  \end{frame}
}

% Let's get started
\begin{document}

\begin{frame}
  \titlepage
\end{frame}



\section{Introduction}


\begin{frame}{Problemas involucrados}
    
\begin{block}{\gls{sat}}
{\small 
\noindent ENTRADA: Un conjunto de cláusulas $C=\left \{c_1, c_2, \dots, c_m \right \}$ sobre un conjunto finito $U$ de variables.

\noindent PREGUNTA: ¿Existe una asignación booleana para $U$, tal que satisfaga todas las cláusulas de $C$? 
}
\end{block}

\begin{block}{\gls{3sat}}
{\small
\noindent ENTRADA: Un conjunto de cláusulas $C=\left \{c_1, c_2, \dots, c_m \right \}$ sobre un conjunto finito $U$ de variables tal que $|c_i|=3$, para $1\le i \le m$.

\noindent PREGUNTA: ¿Existe una asignación booleana para $U$, tal que satisfaga todas las cláusulas de $C$? 
}
\end{block}


\end{frame}


\section{Demostración de NP-completitud}

\begin{frame}{3SAT es NP-completo I}
    
\begin{block}{3SAT $\in \mathcal{NP}$}    
Es fácil comprobar que \gls{3sat}	$\in \mathcal{NP}$, ya que se puede encontrar una algoritmo para una \gls{ndtm} que reconozca el lenguaje $L(\mbox{3SAT},e)$, para un esquema de codificación $e$, en un número de pasos acotado por una función polinomial.

\end{block}

\end{frame}

\begin{frame}{3SAT es NP-completo II}
    
\begin{block}{SAT $\preceq$ 3SAT}    
Sea $U=\left \{u_1, u_2, \dots, u_n  \right\}$  y $C=\left \{c_1, c_2, \dots, c_m \right \}$ una entrada arbitraria de \gls{sat}. Construiremos una conjunto $C'$ de cláusulas de tres literales, basadas en un conjunto $U'$ de variables, tal que $C'$ es satisfactible $\Leftrightarrow$ $C$ es satisfactible.
\end{block}

\end{frame}


\begin{frame}{3SAT es NP-completo III}
    
\begin{block}{SAT $\preceq$ 3SAT}    
\[
U' = U \cup \left( \bigcup_{j=1}^m U'_j\right)
\]

\[
C' = \bigcup_{j=1}^mC_j.
\]\end{block}

\end{frame}


\begin{frame}{3SAT es NP-completo IV}
    
\begin{block}{SAT $\preceq$ 3SAT}    
Supóngase que la cláusula $c_j$ viene definida por los literales $\left \{ {\color{blue}z}_1, {\color{blue}z}_2, \dots, {\color{blue}z}_k  \right\}$, donde cada uno de estos elementos deriva de las variables de $U$. La forma en que construimos los conjuntos  $C'_j$ y $U'_j$ va a depender del valor de $k$.

\vspace{0.5cm}
{\small 
\renewcommand{\arraystretch}{1.8}
\begin{tabular}{@{}L{.10\textwidth}L{.10\textwidth}L{.70\textwidth}@{}} 
Caso 1. & $k = 1$. & $U'_j = \left \{ {\color{magenta}y}_j^1, {\color{magenta}y}_j^2 \right \}$ \\
        &          &$C'_j = \left \{ \left \{  {\color{blue}z}_1, {\color{magenta}y}_j^1, {\color{magenta}y}_j^2  \right \}, \left \{  {\color{blue}z}_1, \bar{{\color{magenta}y}}_j^1, {\color{magenta}y}_j^2  \right \} , \left \{  {\color{blue}z}_1, {\color{magenta}y}_j^1, \bar{{\color{magenta}y}}_j^2  \right \}, \left \{  {\color{blue}z}_1, \bar{{\color{magenta}y}}_j^1, \bar{{\color{magenta}y}}_j^2  \right \}   \right \}$ 	\\
\end{tabular}
}
\end{block}

\end{frame}


\begin{frame}{3SAT es NP-completo IV}
    
\begin{block}{SAT $\preceq$ 3SAT}    
Supóngase que la cláusula $c_j$ viene definida por los literales $\left \{ {\color{blue}z}_1, {\color{blue}z}_2, \dots, {\color{blue}z}_k  \right\}$, donde cada uno de estos elementos deriva de las variables de $U$. La forma en que construimos los conjuntos  $C'_j$ y $U'_j$ va a depender del valor de $k$.

\vspace{0.5cm}
{\small 
\renewcommand{\arraystretch}{1.8}
\begin{tabular}{@{}L{.10\textwidth}L{.10\textwidth}L{.70\textwidth}@{}} 
Caso 2. & $k = 2$. & $U'_j = \left \{ {\color{magenta}y}_j^1 \right \}$ \\
        &          &$C'_j = \left \{ \left \{  {\color{blue}z}_1, {\color{blue}z}_2, {\color{magenta}y}_j^1  \right \}, \left \{  {\color{blue}z}_1, {\color{blue}z}_2, \bar{{\color{magenta}y}}_j^1  \right \}   \right \}$ 	\\
\end{tabular}
}
\end{block}

\end{frame}


\begin{frame}{3SAT es NP-completo IV}
    
\begin{block}{SAT $\preceq$ 3SAT}    
Supóngase que la cláusula $c_j$ viene definida por los literales $\left \{ {\color{blue}z}_1, {\color{blue}z}_2, \dots, {\color{blue}z}_k  \right\}$, donde cada uno de estos elementos deriva de las variables de $U$. La forma en que construimos los conjuntos  $C'_j$ y $U'_j$ va a depender del valor de $k$.

\vspace{0.5cm}
{\small 
\renewcommand{\arraystretch}{1.8}
\begin{tabular}{@{}L{.10\textwidth}L{.10\textwidth}L{.70\textwidth}@{}} 
Caso 3. & $k = 3$. & $U'_j = \emptyset$ \\
        &          &$C'_j = \left \{  c_j  \right \}$ 	\\        
\end{tabular}
}
\end{block}

\end{frame}



\begin{frame}{3SAT es NP-completo IV}
    
\begin{block}{SAT $\preceq$ 3SAT}    
Supóngase que la cláusula $c_j$ viene definida por los literales $\left \{ {\color{blue}z}_1, {\color{blue}z}_2, \dots, {\color{blue}z}_k  \right\}$, donde cada uno de estos elementos deriva de las variables de $U$. La forma en que construimos los conjuntos  $C'_j$ y $U'_j$ va a depender del valor de $k$.

\vspace{0.5cm}
{\small 
\renewcommand{\arraystretch}{1.8}
\begin{tabular}{@{}L{.10\textwidth}L{.10\textwidth}L{.70\textwidth}@{}}      
Caso 4. & $k = 4$. & $U'_j = \left \{ {\color{magenta}y}^i_j | 1 \le i \le k - 3  \right \}$ \\
        &          &$C'_j = \left \{  {\color{blue}z}_1, {\color{blue}z}_2, {\color{magenta}y}_j^1  \right \} \cup \left\{\left \{ \bar{{\color{magenta}y}}_j^i, {\color{blue}z}_{i + 2}, {\color{magenta}y}_j^{i + 1}  \right \} | 1 \le i \le k - 4 \right \} \cup \left \{ \bar{{\color{magenta}y}}_j^{k -3}, {\color{blue}z}_{k -1}, {\color{blue}z}_k  \right \}$ 	\\
\end{tabular}
}
\end{block}

\end{frame}



\begin{frame}{3SAT es NP-completo V}
    
\begin{block}{SAT $\preceq$ 3SAT}    
Tenemos que demostrar que el conjunto de cláusulas $C'$ es satisfactible si y sólo si el conjunto $C$ lo es. 
\begin{itemize}
\item Supóngase primero que $t:U\rightarrow \left \{T,F \right \}$ es una asignación booleana que satisface $C$. 
\item Veremos a continuación que $t$ puede ser extendida a una asignación booleana $t':U'\rightarrow \left \{T,F \right \}$ satisfaciendo $C'$. 
\item Puesto que las variables $U' - U$ están particionadas en conjuntos $U'_j$, y puesto que las variables de cada conjunto $U'_j$ sólo aparecen en las cláusulas $C'_j$, debemos demostrar cómo $t$ puede extenderse para cada conjunto $U'_j$ de forma independiente, en cada caso sólo tenemos que verificar que las cláusulas de $C'_j$ son satisfechas.
\end{itemize}
\end{block}
\end{frame}


\begin{frame}{3SAT es NP-completo VI}
    
\begin{block}{SAT $\preceq$ 3SAT}  
{\small 
\renewcommand{\arraystretch}{1.8}
\begin{tabular}{@{}L{.10\textwidth}L{.10\textwidth}L{.70\textwidth}@{}} 
Caso 1. & $k = 1$. & $U'_j = \left \{ {\color{magenta}y}_j^1, {\color{magenta}y}_j^2 \right \}$ \\
        &          &$C'_j = \left \{ \left \{  {\color{blue}z}_1, {\color{magenta}y}_j^1, {\color{magenta}y}_j^2  \right \}, \left \{  {\color{blue}z}_1, \bar{{\color{magenta}y}}_j^1, {\color{magenta}y}_j^2  \right \} , \left \{  {\color{blue}z}_1, {\color{magenta}y}_j^1, \bar{{\color{magenta}y}}_j^2  \right \}, \left \{  {\color{blue}z}_1, \bar{{\color{magenta}y}}_j^1, \bar{{\color{magenta}y}}_j^2  \right \}   \right \}$ 	\\
Caso 2. & $k = 2$. & $U'_j = \left \{ {\color{magenta}y}_j^1 \right \}$ \\
        &          &$C'_j = \left \{ \left \{  {\color{blue}z}_1, {\color{blue}z}_2, {\color{magenta}y}_j^1  \right \}, \left \{  {\color{blue}z}_1, {\color{blue}z}_2, \bar{{\color{magenta}y}}_j^1  \right \}   \right \}$ 	\\
\end{tabular}
}
 
  
\begin{itemize}
	\item Si se ha construido $U'_j$ sobre los casos 1 o 2, entonces las cláusulas de $C'_j$ ya son satisfechas por $t$, de manera que podemos extender arbitrariamente a $U'_j$, por ejemplo, asignando $t'(y) = T$, para todo $y\in U'_j$. 
\end{itemize}
\end{block}

\end{frame}


\begin{frame}{3SAT es NP-completo VI}
    
\begin{block}{SAT $\preceq$ 3SAT}    


{\small 
\renewcommand{\arraystretch}{1.8}
\begin{tabular}{@{}L{.10\textwidth}L{.10\textwidth}L{.70\textwidth}@{}} 
Caso 3. & $k = 3$. & $U'_j = \emptyset$ \\
        &          &$C'_j = \left \{  c_j  \right \}$ 	\\        
\end{tabular}
}

\begin{itemize}
	\item Si $U'$ se ha construido sobre el caso 3 $U'_j$ es vacío, y por tanto, toda cláusula de $C'$ ya está satisfecha por $t$.
\end{itemize}\end{block}




\end{frame}


\begin{frame}{3SAT es NP-completo VI}
    
\begin{block}{SAT $\preceq$ 3SAT}    

{\small 
\renewcommand{\arraystretch}{1.8}
\begin{tabular}{@{}L{.10\textwidth}L{.10\textwidth}L{.70\textwidth}@{}}      
Caso 4. & $k = 4$. & $U'_j = \left \{ {\color{magenta}y}^i_j | 1 \le i \le k - 3  \right \}$ \\
        &          &$C'_j = \left \{  {\color{blue}z}_1, {\color{blue}z}_2, {\color{magenta}y}_j^1  \right \} \cup \left\{\left \{ \bar{{\color{magenta}y}}_j^i, {\color{blue}z}_{i + 2}, {\color{magenta}y}_j^{i + 1}  \right \} | 1 \le i \le k - 4 \right \} \cup \left \{ \bar{{\color{magenta}y}}_j^{k -3}, {\color{blue}z}_{k -1}, {\color{blue}z}_k  \right \}$ 	\\
\end{tabular}
}

\begin{itemize}
	\item Si se ha construido sobre el caso 4 la cláusula $c_j=\left \{ {\color{blue}z}_1, {\color{blue}z}_2, \dots, {\color{blue}z}_k \right \}$, con $k > 3$. Como $t$ es una asignación booleana que satisface $C$, entonces debe haber un entero $l$ tal que el literal ${\color{blue}z}_l$ tiene un valor verdadero mediante $t$.  
	\begin{itemize} 
	\item Si $l$ es 1 o 2, entonces se asignará $t'({\color{magenta}y}^i_j) = F$ para $1 \le i \le k - 3$.
	\item Si $l$ es $k - 1$ o $k$, entonces se asignará  $t'({\color{magenta}y}^i_j) = T$  para $1 \le i \le k - 3$.
	\item En cualquier otro caso, se asignará  $t'({\color{magenta}y}^i_j) = T$  para $1 \le i \le l - 2$, y $t'({\color{magenta}y}^i_j) = F$  para $l -1 \le i \le k - 3$.
	\end{itemize} 
\end{itemize}\end{block}

\end{frame}

\begin{frame}{Conclusión: 3SAT es NP-completo}
\begin{block}{Conclusión: SAT $\preceq$ 3SAT}   
\begin{itemize}
\item{Todo lo anterior no constituye una prueba de NP-completitud, a no ser que argumentemos que el tamaño del nuevo problema 3SAT está acotado polinómicamente por el tamaño del antiguo problema SAT. \newline \newline Si consideramos cada caso:}
\end{itemize}
\begin{table}[]
\centering

\begin{tabular}{cccc}
\hline
\multicolumn{1}{|c|}{\textbf{Tamaño de cláusula}} & \multicolumn{1}{c|}{\textbf{\# nuevos literales}} & \multicolumn{1}{c|}{\textbf{\# antes}} & \multicolumn{1}{c|}{\textbf{\# después}} \\ \hline
1                                                 & 2                                                 & 1                                      & 12                                       \\
2                                                 & 1                                                 & 2                                      & 6                                        \\
3                                                 & 0                                                 & 3                                      & 3                                        \\
k \textgreater= 4                                 & k-3                                               & k                                      & k + 2(k-3)                              
\end{tabular}
\end{table}
\end{block}

\end{frame}

\begin{frame}{Conclusión: 3SAT es NP-completo}
\begin{block}{3SAT está acotado polinómicamente por SAT}
\begin{itemize}
\item En todos los casos, el tamaño del problema después de ser transformado es a lo sumo 12 veces más grande que el tamaño del problema original.
\end{itemize}
\end{block}

\end{frame}

\end{document}


