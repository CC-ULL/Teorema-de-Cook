\documentclass[a4paper, spanish, utf8]{memoir}

\include{config}
\include{acro}


\title{\Huge Teoría de la NP-completitud}

\author{Adrián Rodriguez Bazaga}
\author{Eleazar Díaz Delgado}
\author{Rudolf Cicko}

\affil{Universidad de La Laguna}


\begin{document}
\maketitle

\begin{abstract}
Your abstract.
\end{abstract}

\chapter{Teorema de Cook}

\section{Introducción}
% \subsection*{Introducción \gls{sat}}

El teorema de Cook también conocido como el teorema de Cook-Levin, se demuestra
el primer problema NP-Completo que permitirá que los demás problemas puedan ser
transformados a él usando una reducción en tiempo polinomial.

El problema usado para probar su NP-completitud es el \gls{sat}.

\section{Problema \gls{sat}}

Demostrar que SAT es un NP es fácil, y se hará a continuación. La parte difícil de la prueba es demostrar  que cualquier lenguaje NP es reducible a \gls{sat} en tiempo polinómico.	

Para ello se construye una reducción polinómica para cada lenguaje A en los NP a SAT. La reducción de $A$ coge una cadena $w$ y produce una fórmula booleana $phi$ que simula la máquina NP para el lenguaje $A$ en la entrada $w$. Si la máquina no acepta, entonces la asignación no satisface $phi$. Por ello, $w$ esta en $A$ si y solo si $phi$ es satisfacible.

Actualmente construir una reducción para trabajar de esta manera es una tarea conceptualmente simple, piensa que se debe enfrentar a muchos detalles. Una formula booleana quizás contenga las operaciones booleanas, AND, OR, NOT y estas operaciones forman la base de la circuitería de los ordenadores electrónicos. De hecho podemos diseñar una fórmula booleana para simular una máquina de Turing.


%% TODO: Definicion del problema esplicacion detallada
El problema de la satisfabilidad booleana, es el problema de determinar si
existe al menos una interpretación que satisfaga una fórmula booleana. Otra forma de verlo es que dado un conjunto de valores booleanos pasados a uno fórmula booleana
hallar aquel conjunto que dé valores verdaderos.

\section{Demostración del Teorema de Cook}

%% TODO: Defninir el problema paso a paso

\noindent ENTRADA: Un conjunto de cláusulas $C=\left \{c_1, c_2, \dots, c_m \right \}$ sobre un conjunto finito $U$ de variables.

\noindent PREGUNTA: ¿Existe una asignación booleana para $U$, tal que satisfaga todas las cláusulas de $C$?


\begin{thm}
  \gls{sat} es $\mathcal{NP}$-completo.
\end{thm}

\begin{proof}
Es fácil comprobar que \gls{sat}	$\in \mathcal{NP}$, ya que se puede encontrar una algoritmo para una \gls{ndtm} que reconozca el lenguaje $L(\mbox{3SAT},e)$, para un esquema de codificación $e$, en un número de pasos acotado por una función polinomial. De esta manera, el primer requerimiento para la prueba de $\mathcal{NP}$-completitud se cumple.

Para el segundo requerimiento debemos bajar a nivel de lenguaje, donde \gls{sat} se representa por un lenguaje $L_{SAT}=L[SAT,e]$ para algún esquema de codificación razonable $e$. Debemos demostrar que para todo lenguaje $L\in \mathcal{NP}$, $L \preceq L_{SAT}$. Los lenguajes de la clase $\mathcal{NP}$ son bastante diversos, se trata de una clase infinitamente grande, por lo que no esperemos presentar una demostración para cada uno de los lenguajes por separado.

No obstante, todo lenguaje de la clase $\mathcal{NP}$ puede describirse de una manera estándar: mediante un programa para una \gls{ndtm} que lo reconozca en tiempo polinomial. Esto nos permitirá trabajar con un programa genérico para una \gls{ndtm} y derivar una transformación polinomial genérica desde el lenguaje que reconozca esta máquina genérica al lenguaje $L_{SAT}$. Esta transformación genérica, cuando se particularice para un programa de una \gls{ndtm} $M$ que reconozca el lenguaje $L_M$, dará la deseada transformación desde $L_M$ a $L_{SAT}$. Así, en esencia, presentaremos una demostración simultánea para todos los lenguajes $L\in \mathcal{NP}$ de que $L \preceq L_{SAT}$.

Para empezar, denotemos por $M$ un programa arbitrario en tiempo polinomial para una \gls{ndtm}, especificado por $\Gamma$, $\Sigma$, $b$, $Q$, $q_0$, $q_Y$, $q_N$, y $\delta$, que reconoce el lenguaje $L=L_M$. Además, sea $p(n)$ un polinomio que acota superiormente la función $T_M(n)$. Sin pérdida de generalidad se asume que $p(n)\ge n$ para todo $n\in \mathbb{Z}^+$. La transformación genérica $f_L$ será derivada en términos de $M$, $\Gamma$, $\Sigma$, $b$, $Q$, $q_0$, $q_Y$, $q_N$, $\delta$ y $p$.

Por razones de conveniencia describiremos $f_L$ como si fuera una función entre el conjunto de las cadenas de $\Sigma$ hasta las entradas del problema \gls{sat}, mas que hasta las cadenas del alfabeto inducido por el esquema de codificación asociado a \gls{sat}, ya que los detalles de la demostración asociados al esquema de codificación pueden ser deducidos fácilmente.

De esta manera, $f_L$ tendrá la propiedad de que para todo $x\in \Sigma^*$, $x\in L$ si, y sólo, $f_L(x)$ es satisfecho por una asignación booleana. La clave de las construcción de $f_L$ está en demostrar cómo un conjunto de cláusulas puede ser usado para comprobar si una entrada $x$ es aceptada por el programa para una \gls{ndtm} $M$, es decir, si $x\in L$.

Si la cadena $x\in \Sigma^*$ es aceptada por $M$, entonces, habrá una secuencia de estados de $M$ que acepte $x$ acotada superiormente por un polinomio $p(n)$, donde $n=|x|$. Dicha secuencia no puede involucrar ninguna celda de la cinta excepto aquellas comprendidas entre $-p(n)$ y $p(n) + 1$ ya que la la cabeza de lectura/escritura comienza en la celda 1, y se mueve una única posición en cada paso. El estado de la computación en cualquier instante puede determinado completamente dando el contenido de las celdas, el estado actual, y la posición de la cabeza de lectura/escritura. Es más, puesto que no se llevarán a cabo más de $p(n)$ pasos durante la ejecución, habrá a los sumo $p(n) + 1$ instantes que deben ser considerados. Esto nos permitirá describir la computación completamente usando sólo un número limitado de variables booleanas y una asignación booleana para las mismas.

A continuación se muestra cómo $f_L$ construye el conjunto de variables a partir de $M$. Se etiquetarán los estados de $Q$ como $q_0, q_1=q_Y, q_2=q_N, q_3, \dots, q_r$, donde $r= |Q|-1$, y los elementos de $\Gamma$ como $s_0 = b, s_1, s_2, \dots, s_v$, donde $v = |\Gamma| -1$. Se crearán tres tipos de variables, cuyo sentido se especifica en la tabla~\ref{tab:cooktab1}.

\begin{table}[!ht]
  \caption{Variables creadas por $f_l$} \label{tab:cooktab1}
  \begin{center}
    {\small
      \renewcommand{\arraystretch}{1.2}
      \begin{tabular}{@{}C{.15\textwidth}C{.25\textwidth}L{.45\textwidth}@{}}
\toprule
{\em Variable} & {\em Rango} &\multicolumn{1}{C{.45\textwidth}}{\em Significado} \\
\midrule
$Q[i,k]$       &

  \begin{minipage}{.24\textwidth} %%
  \centering
    \begin{math} %%
      \begin{array}{c} %%
          0 \le i\le p(n)  \\ %%
          0 \le k \le t %%
        \end{array}%%
    \end{math} %%
  \end{minipage} & En el instante $i$, $M$ está en el estado $q_k$ \\ \\
$H[i,i]$       &

  \begin{minipage}{.24\textwidth} %%
  \centering
    \begin{math} %%
      \begin{array}{c} %%
          0 \le i\le p(n) \\ %%
          -p(n) \le j \le p(n) + 1 %%
        \end{array}%%
    \end{math} %%
  \end{minipage} & En el instante $i$, la cabeza de lectura/escritura está examinando la celda $j$\\ \\
$S[i,j,k]$       &

  \begin{minipage}{.24\textwidth} %%
  \centering
    \begin{math} %%
      \begin{array}{c} %%
          0 \le i\le p(n) \\ %%
           -p(n) \le j \le p(n) + 1 \\ %%
           0 \le k \le v
        \end{array}%%
    \end{math} %%
  \end{minipage} & En el instante $i$, el contenido de la celda $j$ es el símbolo $s_k$\\
\bottomrule
      \end{tabular}
    }
  \end{center}
\end{table}


El cómputo de $M$ devuelve como resultado una asignación booleana para cada una de las variables. Si el programa se detiene antes del tiempo $p(n)$ la configuración permanece constante en todas las ejecuciones posteriores, manteniendo la posición de la cabeza, contenido de la cinta y mismo estado de parada.

Inicialmente el contenido de la cinta consta de la entrada $x$ escrita desde la celda 1 hasta $n$, y el resultado esperado escrito desde la celda -1 hasta la -$|w|$ con el resto de las celdas de la cinta vacías.

Por otro lado, una asignación booleana arbitraria de las variables no tiene por qué corresponder con un cómputo, menos aún con un cómputo de aceptación. 

De acuerdo con una asignación booleana arbitraria, una cinta dada puede contener muchos símbolos al mismo tiempo, la máquina podría estar simultáneamente en varios estados diferentes, y la cabeza de lectura y escritura podría estar en cualquier subconjunto de posiciones $-p(n)$ hasta $p(n)$. La transformación funciona construyendo una colección de cláusulas que implican estas variables, de modo que una asignación booleana es satisfactoria si y sólo si es la asignación booleana inducida por un cómputo de aceptación para $x$ cuya etapa de verificación toma $p(n)$ o menos pasos y cuya cadena adivinada tiene longitud como máximo $p(n)$. Resumiendo, tenemos que: 

$x \in L$ si, y sólo si existe un estado de aceptación en $M$ para $x$, existe un estado de aceptación en $M$ para $x$ en $p(n)$ o menos pasos en la etapa de verificación y con una cadena adivinada $w$ de longitud exactamente $p(n)$ y por último, que se sean aceptadas toda las asignaciones booleanas del conjunto de las cláusulas $f_L(x)$. 

Esto significa que $f_L$ satisface una de las dos condiciones requeridas de una transformación polinomial. La otra condición, que $f_L$ se pueda calcular en tiempo polinomial se verificará fácilmente una vez que hemos completado nuestra descripción de $f_L$.

Las cláusulas en $f_L$ se pueden dividir en seis grupos, cada uno imponiendo un tipo separado de restricción en cualquier asignación de verdad satisfactoria como se muestra en la tabla 1.2.

Es sencillo observar que si todos los grupos de seis cláusulas realizan sus misiones previstas, entonces una asignación booleana satisfactoria tendrá que corresponder al estado de aceptación deseado para x. Así, todo lo que necesitamos mostrar es cómo pueden construirse los grupos de cláusulas que realizan estas misiones.

El grupo $G1$ consiste de las siguientes cláusulas:

\begin{center} { \{$Q[i,0], Q[i,1], \dots, Q[i,r]\}, 0 \le i \le p(n)$ }
\linebreak  { \{$Q[i,j], Q[i,j]\}, 0 \le i \le p(n)$ }
\end{center}


La primeras $p(n)$ + 1 de estas cláusulas puede satisfacerse simultáneamente si y sólo si, para cada vez que $i$, $M$ está en al menos un estado. Las cláusulas restantes $(p(n) + 1)(r + 1)(r / 2)$ pueden satisfacerse simultáneamente si sólo en ningún momento $i$ es $M$ en más de un estado. Así $G1$ cumple su misión. Los grupos $G2$ y $G3$ se construyen de manera similar, y los grupos $G4$ y $G5$ son bastante sencillos, cada uno de ellos consistente sólo en cláusulas literales. La tabla 1.3 da una especificación completa de los cinco primeros grupos. Tenga en cuenta que el número de cláusulas en estos grupos y el número máximo de literales que ocurren en cada cláusula, está acotad por una función polinomial de $n$ (puesto que $r$ y $v$ son constantes determinadas por $M$ y por lo tanto por $L$).


\begin{table}[!ht]
  \caption{Grupos de cláusulas en $f_L(x)$ y las restricciones que imponen} \label{tab:cooktab1}
  \begin{center}
    {\small
      \renewcommand{\arraystretch}{1.2}
      \begin{tabular}{@{}C{.15\textwidth}L{.45\textwidth}@{}}
\toprule
{\em Clase de grupo} &\multicolumn{1}{C{.45\textwidth}}{\em Restricción impuesta} \\
\midrule
$G1$      &
  En el instante $i$, $M$ se encuentra exactamente en un estado \\ 
$G2$       &
  En el instante $i$, la cabeza de lectura/escritura está examinando exactamente una celda. \\ 
$G3$       &
  En el instante $i$, cada celda contiene exactamente un único símbolo de $\Gamma$. \\
$G4$       &
  En el instante 0 la máquina está en su configuración inicial para verificar la entrada $x$.  \\
$G5$        &
  En el instante $p(n)$, $M$ ha cambiado al estado $q_y$ y por lo tanto aceptado $x$.  \\
$G6$       &
  Para cada instante $i$, la configuración de $M$ en el instante $i$ + 1 sigue las indicaciones de la función de transición $\delta$ para la configuración en el instante $i$.    \\ 
\bottomrule
      \end{tabular}
    }
  \end{center}
\end{table}

El último grupo de cláusulas $G6$, que garantiza que cada sucesiva configuración en el cómputo sigue a partir de la anterior en cada paso del programa $M$, es más compleja. En realidad está compuesta de dos subgrupos de cláusulas:  





\end{proof}

\clearpage




%\printglossary  %% no descomentar porque peta todo mal

\bibliographystyle{alpha}
\bibliography{sample}

\end{document}